%% start of file `template.tex'.
%% Copyright 2006-2015 Xavier Danaux (xdanaux@gmail.com).
%
% This work may be distributed and/or modified under the
% conditions of the LaTeX Project Public License version 1.3c,
% available at http://www.latex-project.org/lppl/.


\documentclass[11pt,a4paper,roman]{moderncv}        % possible options include font size ('10pt', '11pt' and '12pt'), paper size ('a4paper', 'letterpaper', 'a5paper', 'legalpaper', 'executivepaper' and 'landscape') and font family ('sans' and 'roman')
\input{defs}

% moderncv themes
\moderncvstyle{classic}                             % style options are 'casual' (default), 'classic', 'banking', 'oldstyle' and 'fancy'

\moderncvcolor{green}                               % color options 'black', 'blue' (default), 'burgundy', 'green', 'grey', 'orange', 'purple' and 'red'
\definecolor{paradisegreen}{HTML}{45A931}
\colorlet{color1}{paradisegreen!80!black}
%\renewcommand{\familydefault}{\sfdefault}         % to set the default font; use '\sfdefault' for the default sans serif font, '\rmdefault' for the default roman one, or any tex font name
%\nopagenumbers{}                                  % uncomment to suppress automatic page numbering for CVs longer than one page

% character encoding
\usepackage[shorthands=off,english]{babel} % package for multilingual support
\RequirePackage{fontspec} % UFT8 fonts for LuaLaTeX
%\usepackage[utf8]{inputenc}                       % if you are not using xelatex ou lualatex, replace by the encoding you are using
%\usepackage{CJKutf8}                              % if you need to use CJK to typeset your resume in Chinese, Japanese or Korean

% adjust the page margins
\usepackage[scale=0.74]{geometry}
%\setlength{\hintscolumnwidth}{3cm}                % if you want to change the width of the column with the dates
%\setlength{\makecvheadnamewidth}{10cm}            % for the 'classic' style, if you want to force the width allocated to your name and avoid line breaks. be careful though, the length is normally calculated to avoid any overlap with your personal info; use this at your own typographical risks...

% personal data
\name{Vladimír}{Štill}
\title{curriculum vit\ae}               % optional, remove the line if not wanted
\address{\mystreet}{635 00 Brno-Bystrc}{Czech Republic} % optional, remove / comment the line if not wanted; the "postcode city" and "country" arguments can be omitted or provided empty
\phone[mobile]{\myphone}                   % optional, remove / comment the line if not wanted; the optional "type" of the phone can be "mobile" (default), "fixed" or "fax"
\email{dev.link@vstill.eu}                        % optional, remove / comment the line if not wanted
\homepage{vstill.eu}    % optional, remove / comment the line if not wanted
\social[linkedin]{vstill}                        % optional, remove / comment the line if not wanted
% \social[twitter]{jdoe}                             % optional, remove / comment the line if not wanted
\social[github]{vlstill}                              % optional, remove / comment the line if not wanted
% \extrainfo{additional information}                 % optional, remove / comment the line if not wanted

\usepackage[ backend=biber
           , style=numeric
           , sorting=ydnt
           , bibencoding=UTF8
           , maxcitenames=3
           , maxbibnames=100
           ]{biblatex}
\DeclareSourcemap{
    \maps[datatype=bibtex, overwrite]{
        \map{
            \step[fieldset=editor, null]
            \step[fieldset=language, null]
        }
    }
}
\addbibresource{papers.bib}

% \renewbibmacro*{date}{}
\renewbibmacro*{date+extrayear}{}
\renewbibmacro*{issue+date}{}
\newcommand*{\bibyear}{}

\defbibenvironment{bibliography}
  {\list
     {\iffieldequals{year}{\bibyear}
        {}
        {\printfield{year}%
         \savefield{year}{\bibyear}}}
     {\setlength{\topsep}{0pt}% layout parameters based on moderncvstyleclassic.sty
      \setlength{\labelwidth}{\hintscolumnwidth}%
      \setlength{\labelsep}{\separatorcolumnwidth}%
      \setlength{\itemsep}{\bibitemsep}%
      \leftmargin\labelwidth%
      \advance\leftmargin\labelsep}%
      \emergencystretch 3em\clubpenalty4000\widowpenalty4000}
  {\endlist}
  {\item}

\usepackage[plainpages=false,   % get the page numbering correctly
            pdfpagelabels,      % write arabic labels to all pages
            unicode,               % allow unicode characters in links
            colorlinks=true,    % use colored links instead of boxed
            linkcolor={color1},
            urlcolor={color1}
            ]{hyperref}

\usepackage{xspace}
\newcommand{\divine}{\mbox{DIVINE}\xspace}
\newcommand{\lart}{\mbox{\textsf{LART}}\xspace}
\newcommand{\llvm}{\textsf{LLVM}\xspace}
\newcommand{\ltl}{LTL\xspace}
\usepackage{enumitem}

\begin{document}
\makecvtitle

\section{Education}
\cventry{2016--2020}{Masaryk University}{}{Brno, Czech Republic}{}{%
PhD thesis: \emph{Analysis of Parallel C++ Programs}%
\newline{}%
Field of Study: FI PST Computer Systems and Technologies%
}
\cventry{2018}{Masaryk University}{}{Brno, Czech Republic}{}{%
Advanced Master's state examination (RNDr.)%
\newline{}%
Thesis: \textit{Memory-Model-Aware Analysis of Parallel Programs}%
}
\cventry{2013--2016}{Masaryk University}{}{Brno, Czech Republic}{}{%
Master thesis: \textit{\llvm{} Transformations for Model Checking}%
\newline{}%
Field of study: FI PDS Parallel and Distributed Systems%
}
\cventry{2010--2013}{Masaryk University}{}{Brno, Czech Republic}{}{%
Bachelor thesis: \textit{State space compression for the \divine model checker}%
\newline{}%
Field of study: FI PSK Computer Networks and Communication%
}

\section{Work Experience}
\cventry{2020--now}{Masaryk University}{}{Brno, Czech Republic}{}{
	Lecturer in programming at Faculty of Informatics (full-time).
}
\cventry{2016–2020}{Masaryk University}{}{Brno, Czech Republic}{}{
        PhD Researcher in the topic of \emph{Analysis of Parallel C++ Programs}, supported in part by Red Hat Czech Republic.
}
%\cventry{2011–2020}{Masaryk University}{}{Brno, Czech Republic}{}{
%        Teaching assistant (part-time tutor).
%}

\section{Research}
\cventry{2012--2020}{Parallel \& Distributed Systems Laboratory}{}{Masaryk University, Brno, Czech Republic}{}{
Active lab member, developer of the \divine model checker.
My PhD work on \divine was in part supported by Red Hat Czech Republic.
My main topic was verification of parallel C and C++ programs, including research in handling advanced language features such as exceptions, analysis of relaxed memory models, and parallel program termination.\\[\smallskipamount]
I was participating in the following grants:
\begin{itemize}[itemsep=0pt, topsep=\smallskipamount]
  \item Czech Science Foundation (GAČR) grants GA18-02177S, GA15-08772S, GA P202\slash{}11\slash{}0312,
  \item Technology Agency of the Czech Republic (TAČR) grant TH04010192,
  \item Grants of the FI Dean's Programme in 2013–2015.
  %: MUNI\slash{}33\slash{}05\slash{}2013, MUNI\slash{}33\slash{}13\slash{}2014, and MUNI\slash{}33\slash{}15\slash{}2015.
\end{itemize}
}

\section{Other Work Experience}
\cventry{2015--2020}{Masaryk University}{}{Brno, Czech Republic}{}{
    Chairman of the Students' Chamber of the Academic Senate of Faculty of Informatics.
    \newline{}
    Member of the Dean's Board.
}
\cventry{2018--2020}{Masaryk University}{}{Brno, Czech Republic}{}{
    Member of the Students' Chamber of the Academic Senate of Masaryk University.
    \newline{}
    Member of the Doctoral Studies Committee of the Academic Senate of Masaryk University.
}

\section{Selected Software Projects}
\cventry{2012--2020}{\divine}{}{}{\emph{Developer \& Researcher}}{
    A software model checker developed in the ParaDiSe Laboratory, written mostly in C++.\\
}
\cventry{2014--now}{hsExprTest}{}{}{\emph{Lead Developer}}{
    A framework for automatic testing of students' assignment using
    questionnaires or submission folders in Information System of Masaryk University and a tool for checking of
    Haskell assignments against teacher-provided solutions or tests.
    \begin{itemize}
        \item Since 2014 used in IB015 Non-Imperative Programming
        \item Since 2018 used in IB002 Algorithms and data structures I
        \item In 2018 used in IB113 Introduction to Programming and Algorithms
        \item Since 2019 used in IB111 Foundations of Programming
        \item Since 2020 used in IB005 Formal Languages and Automata
        \item Since 2020 used in IB016 Seminar on Functional Programming
    \end{itemize}
    The generic framework is written in Python (with previous versions in C++
    and Haskell), the Haskell support is written mainly in Haskell.
    Support for courses other then IB015 and IB016 was provided by teachers of
    the given courses.
}

\section{Teaching Experience}
\cventry{2011--now}{Masaryk University}{}{Brno, Czech Republic}{}{}
    % beware of nasty hack here to allow page breaks inside the description -- the description is not in the enumitem
    \begingroup\small
    \begin{description}[font=\itshape, topsep=0pt,
                        leftmargin=\hintscolumnwidth+\separatorcolumnwidth+\labelwidth+2\labelsep,
                        labelindent=\hintscolumnwidth+\separatorcolumnwidth]
        \item[IB015 Non-inperative Programming] head of the team responsible for homework creation and automatic grading, 2014--now; head of seminars, 2020--now; preparation and grading of final exams, 2016--now; seminar tutor, 2011--2016, 2020--now; occasionally substitute lecturer.

            In 2014 I was part of the team which re-designed the course seminars and created materials and homework for them.

            In 2019 I was part of the team which re-designed the course once more due to changes in the FI study programmes.

        \item[PV264 Advanced Programming in C++] seminar tutor, homework creator, occasionally substitute lecturer, and lecture and seminar material co-author, 2017--now.

            In 2016 I was part of the team which created this course and materials for it.

        \item[IB016 Seminar on Functional Programming] seminar tutor and teaching material co-author, 2015--2019, 2021--now.

            In 2016 I was part of the team which re-designed this course.

        \item[PB173 Domain specific development in C/C++]
			Guarantor, and seminar tutor, 2021--now.

			Seminar tutor for the topic of Modern C++, 2014--2016 (I was part of the team which created this topic for PB173, this later lead to creation of PV264).

        \item[PB161 C++ Programming] lecturer of the first three lectures, 2020; seminar tutor, 2021–now; occasionally substitute lecturer.
		
		\item[IB111 Foundations of Programming] seminar tutor, 2020--now.

		\item[PB152cv Operating Systems – Practicals] seminar tutor, 2021--now.

		\item[PV248 Python] seminar tutor, 2020--now.

		\item[PA193 Secure coding principles and practices] lecturer \& tutor for two topics, 2021.


        \item[IB005 Formal Languages and Automata] head of the team responsible for homework creation, 2020--now.

		\item[PB006 Principles of Programming Languages and OOP] responsible for homework creation, 2020.


        \item[IB102 Automata, Grammars, and Complexity] homework creation and grading, 2012; head of the team responsible for homework creation and grading, 2013--2019; seminar tutor, 2013--2014.

        \item[PB071 Introduction to the C language] seminar tutor in autumn 2012.
    \end{description}
    \endgroup

\section{Other Experience}

% \cventry{2019--now}{Instruktoři Brno}{}{}{}{
% \textcolor{red}{TODO}
% }
\cventry{2014--now}{Nordic Animals Association}{}{Masaryk University}{}{
Member of the community helping with organization of educational and entertainment activities for pre-university and university students.
}
\cventry{spring 2014}{Norwegian University of Science and Technology}{}{Trondheim, Norway}{}{
Erasmus stay for one semester.
}

\section{Language Skills}
\cvlanguage{Czech}{native language}{}
\cvlanguage{English}{fluent}{}
%\cvlanguage{German}{basics}{}
%\cvlanguage{Norwegian}{basics}{}


\section{Programming Language Skills}

\cvlanguage{C++}{advanced knowledge}{}
\cvlanguage{Haskell, Python, C}{very good knowledge}{}
\cvlanguage{Perl, Prolog, C\#}{average knowledge}{}

\clearpage
\nocite{*}
\printbibliography[title={Publications}]

\end{document}


%% end of file `template.tex'.
